\documentclass[11pt,a4paper]{ivoa}
\input tthdefs

\title{Virtual Observatory and High Energy Astrophysics}

% see draft note here:

% see ivoatexDoc for what group names to use here; use \ivoagroup[IG] for
% interest groups.
\ivoagroup{DM}

\author{HE club}

\editor{Mathieu Servillat}

% \previousversion[????URL????]{????Concise Document Label????}
\previousversion{This is the first public release}

\usepackage{longtable}
%\usepackage{booktabs} % For prettier tables
\usepackage{lscape}
%\usepackage{minted}

\setlength {\marginparwidth }{2cm}
\usepackage{todonotes}

\begin{document}

\begin{abstract}
Virtual Observatory and High Energy Astrophysics
\end{abstract}


\section*{Acknowledgments}

We acknowledge support from the ESCAPE project funded by the EU Horizon 2020 research and innovation program (Grant Agreement n.824064).
Additional funding was provided by the INSU (Action Sp\'ecifique Observatoire Virtuel, ASOV), the Action F\'ed\'eratrice
CTA and the Action Pluriannuelle Incitatrice Astrophysioque des processus de Hautes \'Energies at the Observatoire de
Paris, and the Paris Astronomical Data Centre (PADC).

\section*{Conformance-related definitions}

The words ``MUST'', ``SHALL'', ``SHOULD'', ``MAY'', ``RECOMMENDED'', and
``OPTIONAL'' (in upper or lower case) used in this document are to be
interpreted as described in IETF standard RFC2119 \citep{std:RFC2119}.

The \emph{Virtual Observatory (VO)} is a
general term for a collection of federated resources that can be used
to conduct astronomical research, education, and outreach.
The \href{https://www.ivoa.net}{International
Virtual Observatory Alliance (IVOA)} is a global
collaboration of separately funded projects to develop standards and
infrastructure that enable VO applications.


\section{Introduction}

% We should introduce the purpose of the note in distribution and access of event list data products. Science cases should be focused to highlight that.

High Energy (HE) astronomy typically includes X-ray astronomy, gamma-ray astronomy of the GeV range (HE), the TeV range
(very high energy, VHE) up to the ultra high energy (UHE) above 100 TeV, the VHE neutrino
astronomy, and studies of cosmic rays. This domain is now sufficiently developed to provide high level data such as catalogs, images, including full-sky surveys for some missions, and sources properties in the shape of spectra and time series.
Such high level, HE observations have been included in the VO, via data access endpoints provided by observatories or by agencies and indexed in the VO Registry.
%Some high-energy (HE) data is already available via the VO. Images, time series, and spectra may be described with Obscore and access.

However, after browsing this data, users may want to download lower level data and reapply data reduction steps relevant to their Science objectives. A common scenario is to download HE "event" lists, i.e. lists of detected events on a HE detector, that are expected to be detection of particles (e.g. a HE photon), and the corresponding calibration files, including Instrument Response Functions (IRFs). The findability and accessibility of these data via the VO is the focus of this note.

We first report typical use cases for data access and analysis of data from current HE observatories. From those use
cases, we note that some existing IVOA recommendations are of interest to the domain. They should be further explored
by HE observatories. We then discuss how standards could evolve to better integrate specific aspects of HE data, and if
new standards should be developed.

\subsection{Objectives of the document}

The main objective of the document is to analyse how HE data can be better integrated to the VO.

We  first identify and expose the specificities of HE data from several HE observatories. Then we intend to illustrate how HE data is or can be handled using current IVOA standards. Finally, we explore several topics that could lead to HE specific recommandations.

A related objective is to provide a context and a list of topics to be further discussed within the IVOA by a dedicated HE Interest Group.


\subsection{Scope of the document}

This document mainly focuses on HE data discovery through the VO, with the identification of common use cases in the HE astrophysics domain, which provides an insight of the specific metadata to be expose through the VO for HE data.

Some current existing IVOA recommendations is discussed in this document within the HE context and they will be in-depth
studied in the High Energy Interest Group (HEIG).



% \subsection{Role within the VO Architecture}

% \begin{figure}
% \centering

% % As of ivoatex 1.2, the architecture diagram is generated by ivoatex in
% % SVG; copy ivoatex/archdiag-full.xml to role_diagram.xml and throw out
% % all lines not relevant to your standard.
% % Notes don't generally need this.  If you don't copy role_diagram.xml,
% % you must remove role_diagram.pdf from SOURCES in the Makefile.

% \includegraphics[width=0.9\textwidth]{role_diagram.pdf}
% \caption{Architecture diagram for this document}
% \label{fig:archdiag}
% \end{figure}

% Fig.~\ref{fig:archdiag} shows the role this document plays within the
% IVOA architecture \citep{2010ivoa.rept.1123A}.


\section{High Energy observatories and experiments}
%XMM use case scenario
%Données attachées ? data link?

There are various observatories, either from ground, space or deep-sea based, that distribute high-energy data with
different level of involvement in the VO. We list here the observatories currently represented in the VO HE group.
There are also other observatories that are connected to the VO in some way, and may join the group discussions at IVOA.


\subsection{Gamma-ray programs}

\subsubsection{H.E.S.S}
\label{sec:hess}

The High Energy Steresopic System (H.E.S.S.) experiment is an array of Imaging Atmospheric Cherenkov Telescopes (IACT)
located in Namibia that investigates cosmic very high energies (VHE) gamma rays in the energy range from 10s of GeV to
100 of TeV. It is constituted of four telescopes officially inaugurated in 2004, and a much larger fifth telescope
operational since 2012, extending the energy coverage towards lower energies and further improving sensitivity.

The H.E.S.S. collaboration operates the telescopes as a private experiment and publishes mainly high level data,
i.e. images, time series and spectra in scientific publications after dedicated analyses. Using complex algorithms,
private software process the raw data by applying calibration, reconstructing event properties from their Cherenkov
images and purifying the event list by removing as much as possible events induced by atmospheric cosmic rays (CRs). Even
after this purification, events are largely generated by CRs and statistical analyses are required to derive
the astrophysical source properties. Models of background due to the remaining CRs
(generally generated from real observations) are used with the gamma-ray IRFs (PSF, Energy Dispersion, Collection Area)
that are generated by extensive Monte Carlo simulations. These 4 IRFs (background, PSF, Edisp, CollArea) are computed
for each observation of $\sim$~30min and are valid for the field of view. They depend on true energies, positions in the
field of view and sometimes from event classification types. The derivation of astrophysical quantities from
the event lists are now using open libraries, in particular the reference library Gammapy \citep{gammapy:2023}.
%% Need to describe the IRFs like for Chandra?

In September 2018, the H.E.S.S. collaboration has, for the first time and unique time, released a small subset of its
archival data using the GADF format (see~\ref{sec:GADF}) serialized into the Flexible Image Transport System (FITS) format,
an open file format widely used in astronomy. The release consists of Cherenkov event-lists and IRFs for observations of
various well-known gamma-ray sources \citep{hess-zenodo.1421098}.

This test data collection has been registered in the VO via a TAP service hosted at the Observatoire de Paris, with a
tentative ObsCore description of each dataset. We hope that, in the future, the H.E.S.S. legacy archive will be published
in a similar way and accessible through the VO.

\subsubsection{CTAO}
\label{sec:ctao}

The Cherenkov Telescope Array Observatory (CTAO) is the next generation ground-based IACT instrument for gamma-ray astronomy
at very high energies. With tens of telescopes located in the northern (La Palma, Canary Island)
and southern (Chili) hemispheres, CTAO will be the  first open ground-based VHE gamma-ray observatory and the world’s
largest and most sensitive instrument to study  high-energy phenomena in the Universe. Built on the technology of current
generation ground-based gamma-ray detectors (e.g. H.E.S.S., MAGIC and VERITAS), CTAO will be between five and 10 times
more sensitive and have unprecedented accuracy  in its detection of VHE gamma rays.

CTAO will distribute data as an open observatory, for the first time in this domain, with calls for proposals and
publicly released data after a proprietary period. CTAO will ensure that the provided data will be FAIR: Findable,
Accessible, Interoperable and Reusable, by following the FAIR Principles for data management \citep{Wilkinson2016}.
In particular, because of the complex data processing and reconstruction steps, the provision of provenance metadata
for CTAO data has been a driver for the development of a provenance standard in astronomy.

CTAO will also ensure VO compatibility of the distributed data and access systems. CTAO participated to the ESCAPE
European Project, and is now part of the ESCAPE Open Collaboration to face common challenges for Research Infrastructures
in the context of cloud computing, including data analysis and distribution.

A focus of CTAO is to distribute in this context their Data Level 3 (DL3) datasets, that correspond to lists of Cherenkov
events detected by the telescopes along with the proper IRFs. CTAO is planning an internal and a public Science Data
Challenges, which represent opportunities to build "VO inside" solutions.
%% Need to describe the IRFs like for Chandra?

The CTAO observatory is complementary to other gamma-ray instruments observing the sky up to ultra high energies (ie PeV).
Detecting directly from ground secondary charged particles of extensive air showers initiated by gamma rays, Water
Cherenkov Detectors (WCD) survey the whole observable sky above the TeV/tens of TeV energy range. The HAWC and LHAASO
detectors are running in the northern hemisphere and the future SWGO observatory will be installed in the southern
hemisphere. Such instruments have similar high-level data structures and it has been already demonstrated that joined
analyses with Gammapy of data from IACTs and WCDs using the GADF format are very powerful \citep{2022A&A...667A..36A}.

\subsection{X-ray programs}

\subsubsection{Chandra}\label{sec:chandra}

Part of NASA's fleet of ``Great Observatories'', the Chandra X-ray Observatory (CXO) was launched in 1999 to observe
the soft X-ray universe in the 0.1 to 10 keV energy band. Chandra is a guest observer, pointed-observation mission and
obtains roughly 800 observations per year using the Advanced CCD Imaging Spectrometer (ACIS) and High Resolution Camera
(HRC) instruments. Chandra provides high angular resolution with a sub-arcsecond on-axis point spread function (PSF),
a field of view up to several hundred square arcminutes, and a low instrumental background. The Chandra PSF varies with
X-ray energy and significantly with off-axis angle, increasing to R50 $\sim$25 arcsec at the edge of the field of view.
A pair of transmission gratings can be inserted into the X-ray beam to provide dispersed spectra with E/DeltaE $\sim$1000
for bright sources. The Chandra spacecraft normally dithers in a Lissajous pattern on the sky while taking data, and
this motion must be removed from the time-resolved X-ray event lists when constructing X-ray images using the motion
of optical guide stars tracked by the Aspect camera.

% Are the analysis step description made below in necessary? for the homogenity between instruments
The Chandra X-ray Center (CXC) processes the spacecraft data through a set of Standard Data Processing Level 0 through
Level 2 pipelines. These pipelines perform numerous steps including decommutating the telemetry data,
applying instrument calibrations (e.g., detector geometric, time- dependent gain, and CCD charge transfer efficiency
(CTI) corrections, bad and hot pixel flagging), computing and applying the time-resolved Aspect solution to de-dither
the motion of the telescope, identifying good time intervals (GTIs), and finally filtering out bad times and X-ray events
with bad status. All data products are archived in the Chandra Data Archive (CDA) in FITS format following HEASARC
OGIP standards;  see also \S~\ref{sec:ogip}. The CDA manages the proprietary data period (currently 6 months, after
which the data become public) and provides dedicated interactive and IVOA-compliant interfaces to locate and download
datasets.

The CXC also provides the Chandra Source Catalog, which in the latest release (2.1) includes data for $\sim$407K unique
X-ray sources on the sky and more than 2.1 million individual detections and photometric upper limits. For each X-ray
source and detection, the catalog provides a detailed set of more than 100 tabulated positional, spatial, photometric,
spectral, and temporal properties. An extensive selection of individual observation, stacked-observation, detection
region, and master source FITS data products (e.g., RMFs, ARFs, PSFs, spectra, light curves, aperture photometry MPDFs)
are also provided that are directly usable for further detailed scientific analysis.

% According to https://heasarc.gsfc.nasa.gov/docs/heasarc/caldb/docs/memos/cal_gen_92_002/cal_gen_92_002.html#tth_sEc2.1,
% RMF, ARF and PSF does not depend on spectral models
Finally, the CXC distributes the CIAO data analysis package to allow users to recalibrate and analyze their data. A key
aspect of CIAO is to provide users the ability to create instrument responses (RMFs, ARFs, PSFs, etc) for their
observations. The Sherpa modeling and fitting package supports N-dimensional model fitting and optimization in Python,
and supports advanced Bayesian Markov chain Monte Carlo analyses.

\subsubsection{XMM-Newton}

The European Space Agency's (ESA) X-ray Multi-Mirror Mission (XMM-Newton) was launched in 1999. XMM-Newton is ESA's
second cornerstone of the Horizon 2000 Science Program. It carries 3 high throughput X-ray telescopes with an
unprecedented effective area, and an optical monitor, dedicated to the study of celestial X-ray sources.

\todo[inline]{To be completed: XMM catalogs, data... and VO access.}


\subsubsection{SVOM}

The SVOM mission (Space-based multi-band astronomical Variable Objects Monitor) is a Franco-Chinese mission dedicated
to the study of the most distant explosions of stars, the gamma-ray bursts. It is to be launched in 2024.

\todo[inline]{To be completed}



\subsection{KM3Net and neutrino detection}

The KM3NeT neutrino detectors are arrays of water-based Cherenkov detectors currently under construction in the deep
Mediterranean Sea. With its two sites off the French and Italian coasts, the KM3NeT collaboration aims at single particle
neutrino detection for neutrino physics with the more densely instrumented ORCA detector in the GeV to TeV range, and
VHE astrophysics with the ARCA detector in the TeV range and above.

Using Earth as a shield from atmospheric particle interference by searching for upgoing particle tracks in the detectors,
the measurement of astrophysical neutrinos can be performed almost continuously for a wide field of view that covers the
full visible sky. For these particle events, extensive Monte Carlo simulations are performed to evaluate the
statistical significance towards the various theoretical assumptions for galactic or cosmic neutrino signals.

During the construction phase, the KM3NeT collaboration develops its interfaces for open science and builds on the data
gathered by its predecessor ANTARES, from which neutrino event lists have already been published on the KM3NeT VO server
as TAP service. However, {\bf reproducibility of the searches for point-like sources -> computation of astrophysical quantities}
require information derived from  simulations like background estimate, PSF and detector acceptance which require linking
to the actual event list and interpolation for a given observation.

With multiple detectors targeting high-energy neutrinos like IceCube, ANTARES, KM3NeT, Baikal and future projects, the
chance to detect a significant amount of cosmic and galactic neutrinos increases, requiring an integrated approach to
link event lists with instrument responses and to correctly interpret observation time and flux expectations.

% mireille : what is specific for the community in terms of data interpretation and computation steps

\section{Common practices in the High Energy community}
\label{sec:vhespec}

\subsection{Data flow specificities}

\subsubsection{Event-counting}

Observations of the Universe at high energies are based on techniques that are radically different compared to the optical, or radio domain. HE observatories are generally designed to detect particles, e.g. individual photons, cosmic-rays, or neutrinos, with the ability to estimate several characteristics of those particles. This technique is generally named \textbf{event counting}, where an event has some probability of being due to the interaction of an astronomical particle with the detectors.

The data corresponding to an \textbf{event} is first an instrumental signal, which is then calibrated and processed to estimate event characteristics such as a time of arrival, coordinates on the sky, and the energy proxy associated to the event. Several other intermediate and qualifying characteristics can be associated to a detected event.

When observing during an interval of time, the data collected is a list of the detected events, named an \textbf{event list} in the HE domain, and event-list in this document.

%HE projects already have data formats in use to transport the results of observations together with the necessary instrument response files.

%Such response files depend on the way raw event lists are combined together; they are essential for the calibration steps that will help to produce calibrated event-lists in position, time and energy.


\subsubsection{Data levels}\label{sec:datalevels}

After detection of events, data processing steps are applied to generate data products. We typically distinguish at least 3 main data levels.

\begin{itemize}
    \item[1] An event-list with calibrated temporal and spatial characteristics, e.g. sky coordinates for a given epoch, event arrival time with time reference, and a proxy for particle energy.
    \item[2] Binned and/or filtered event list suitable for preparation of science images, spectra or light-curves.  For some instruments, corresponding instrument responses associated with the event-list, calculated but not yet applied (e.g, exposure maps, sensitivity maps, spectral responses).
    \item[3] Calibrated maps, or spectral energy distributions for a source, or light-curves in physical units.
    \item[4] An additional data level may correspond to catalogs, e.g. a source catalog pointing to several data products (e.g. collection of L3 products) with each one corresponding to a source.
\end{itemize}

However, the definitions of these data levels can vary significantly from facility to facility, and may not map directly to separate ObsCore calib\_levels.

For example, in the VHE Cherenkov astronomy domain (e.g. CTA), the data levels listed above are labelled DL3\footnote{events being reconstructed, lower level data is specific this domain (DL0--DL2).} to DL5.  However, for Chandra X-ray data, the first two levels correspond  to L1 and L2 data products (excluding the responses), while transmission-grating data products are designated L1.5 and source catalog and associated data products are all designated L3.


\subsubsection{Background signal}

Observations in HE may contain a high background component, that may be due to instrument noises, or to unresolved astrophysical sources, emission from extended regions or other terrestrial sources producing particles similar to the signal. The characterization and estimation of this background may be particularly important to then apply corrections during the analysis of a source signal.

In the VHE domain with the IACT, WCD and neutrino techniques, the background is created by cosmic-ray induced events. The case of unresolved astrophysical sources, emission from extended regions are treated as a model of a gamma-ray or neutrino emission.  In the X-ray domain, contributions to background can include an instrumental component, the local radiation environment (i.e. space weather) which can change dynamically, and may include the cosmological background due to unresolved astrophysical sources, depending on the spatial resolution of the instrument.


\subsubsection{Time intervals}

Depending on the stability of the instruments and observing conditions, a HE observation can be decomposed into several intervals of time that will be further analysed.
For example, Stable Time Intervals (STI) are defined in Cherenkov astronomy to characterize periods of time during which the instrument response is stable. In the X-ray domain, Good Time Intervals (GTI) are computed to exclude time periods where data are missing or invalid, and may be used to reject periods impacted by high radiation, e.g. due to space weather. In contrast, for neutrino physics, relevant observation periods can cover up to several years due to the low statistics of the expected signal and a continuous observational coverage of the full field of view.


\subsubsection{Instrument Response Functions}

Though an event-list can contain calibrated physical values, typically the data still has to be corrected for the photometric, spectral, spatial, and/or temporal responses of the instruments used to yield scientifically interpretable information. The IRFs provide mappings between the physical properties of the source and the observables, and so enable estimation of the former (such as the real flux of particles arriving at the instrument, the spectral distribution of the particle flux, and the temporal variability and morphology of the source).  Note that the small number of particles detected in many types of HE observations (i.e., Poisson regime) imply that the IRFs may not be directly invertible, so that techniques such as forward fitting are needed to estimate the physical properties of the source from the observables.  Depending on the instrument, this may imply that some IRFs cannot be easily pre-computed because they may depend on details (e.g. the shape of the source model spectrum) of the scientific analysis to be performed.

The instrumental responses typically vary with the true energy of the event, the progenitor of the event (photon vs residual cosmic ray background), as well as the position in the field of view.
A further complication of ground-based detectors like IACTs and WCTs is that the instrumental responses also vary with:
\begin{itemize}
  \item The horizontal coordinates of the atmosphere, i.e. the response to a photon at low elevation is different from that at zenith due to a larger air column density, and different azimuths are affected by different magnetic field strengths and directions that modify the air-shower properties.
  \item The atmosphere density, which can have an effect on the response that changes throughout a year, depending on the site of observation.
  \item The brightness of the sky (for IACTs), i.e. the response is worse when the moon is up, or when there is a strong nigh-sky-background level from e.g. the Milky Way or Zodiacal light.
\end{itemize}
Since these are not aligned with a sky coordinate system, field-rotation during an observation must also be taken into account.
Therefore the treatment of the temporal variation of IRFs is important, and is often taken into account in analysis by averaging over some short time period, such as the duration of the observation, or intervals within.

\subsubsection{Granularity of data products}

In order to allow for multi-wavelength data discovery of HE data products and compare observations across different regimes, it seems appropriate to distribute the metadata in the VO ecosystem together with an access link to the data file in community format for finer analysis.

Where feasible, the efficient granularity for distributing HE data products seems to be the full combination of data and associated IRFs. Depending on the instrument, some IRFs may need to be (re-)computed by a service or tool after parameter selection by the user, so inclusion of additional files that are required for these steps should be included in the package where appropriate.

% mir already mentionned above why we should consider IRF
%The coverage information, i.e. how the data product spans on the sky coordinates, and along time and energy axis, is an important criterium for data selection. In the case of HE observations, these parameters vary depending on the selected good time intervals.
% to be developed

The event-list dataset is generally stored as a table, with one row per candidate detection (event) and several columns for the observed and/or estimated physical parameters (e.g. arrival time, position (on detector or in the sky), energy or pulse height, and additional properties such as errors or flags that are project-dependent) that can vary with data level.

The list of columns present in the event-list is for example described in the data format in use in the HE domain, such as OGIP or GADF as introduced below. The data formats in use generally describe the event-list data together with the IRFs and other relevant information, such as: Stable or Good Time Interval, Effective Area, Energy Dispersion, Point Spread Function, Background,...

%From an observation file, the event data can lead to several data products results, depending on the parameter selections induced from the knowledge of the IRF.


\subsection{Work flow specificities}

\subsubsection{Event selection}

When processing an event-list, it is important to perform an optimal {selection of the events} that are more likely to be due to the incident particles expected. This selection may depend on the source targeted or on the science objectives. The selection can be performed on the event characteristics, e.g. time, energy or more specific indicators (patterns, shape...)


\subsubsection{Assumptions and probabilistic approach}

In order to produce advanced data products like light curves or spectra, {assumptions} about the kind of particles, noise, source type and its expected energy distribution must be introduced. This is one of the main driver for enabling a full and well described access to event-list data, as scientific analyses generally start at this data level.


\subsection{Data formats}
\label{sec:data_formats}

\subsubsection{{OGIP}}\label{sec:ogip}

NASA's HEASARC FITS Working Group was part of the Office of Guest Investigator Programs, or OGIP, and created in the 1990's the multi-mission standards for the format of FITS data files in NASA high-energy astrophysics. Those so-called OGIP  recommendations\footnote{\url{https://heasarc.gsfc.nasa.gov/docs/heasarc/ofwg/ofwg_recomm.html}} include standards on keyword usage in metadata, on the storage of spatial, temporal, and spectral (energy) information, and representation of response functions, etc.  These standards predate the IVOA but include such VO concepts as data models, vocabularies, provenance, as well as the corresponding FITS serialization specification.

The purpose of these standards was to allow all mission data archived by the HEASARC to be stored in the same data format and be readable by the same software tools. \S~\ref{sec:chandra} above, for example, describes the Chandra mission products, but many other smaller projects do so as well.  Because of the OGIP standards, the same software tools can be used on all of the high-energy mission data that follow them.  There are now some thirty plus different mission datasets archived by NASA following these standards and different software tools that can analyze any of them.

Now that the IVOA is defining data models for spectra and time series, we should be careful to include the existing OGIP standards as special cases of what are developed to be more general standards for all of astronomy.


\subsubsection{GADF and VODF}
\label{sec:GADF}

The data formats for gamma-ray astronomy\footnote{\url{https://gamma-astro-data-formats.readthedocs.io/}} (GADF) is a community-driven initiative for the definition of a common and open high-level data format for gamma-ray instruments \citep{2017AIPC.1792g0006D,2021-DF} starting at the reconstructed event level. GADF is based partially on the OGIP standards and is specialised for Very High Energy data. It was originally developed in 2011 for CTAO during it's prototyping phase, and was further tested on data from the HESS telescope array, where it is now used as the standard for science analysis. The project was made open-source in 2016, and became the base format for the \emph{gammapy} software.

The Very-high-energy Open Data Format\footnote{\url{https://vodf.readthedocs.io/}} (VODF), will build upon and be the successor to GADF. It is intended to address some of the short-comings of the GADF format, to provide a properly documented and consistent data model, to cover use cases of both  Very-High-Energy (VHE) gamma-ray and neutrino astronomy, and to provide more support for validation and versioning. VODF will provide a standard set of file formats for data starting at the reconstructed event level (Level 1 in \autoref{sec:datalevels}) as well as higher-level products such as N-dimensional binned data cubes (including sky images, light curves, and spectra) and source catalogues (Levels 2-4 in \autoref{sec:datalevels}. With these standards, common science tools can be used to analyze data from multiple high-energy instruments, including facilitating the ability to do combined likelihood fits of models across a wide energy range directly from events or binned products. VODF aims to follow or be compatible with existing IVOA standards as much as possible.

\subsection{Tools for data extraction and visualisation}
\label{sec:tools}

HE data is particularly complex and diverse at lower levels. It is common to find specific tools to process the data for a given facility, e.g. CIAO for Chandra, SAS fro XMM-Newton, of Gammapy for gamma-ray data, with a particular focus on Cherenkov data as foreseen for CTA.

Those tools can generally handle data from several other obsevatories, that have some level of commonalities.

Several other HE software are build to handle the existing data format standards, hence enabling mutli-instrument studies, e.g. XSpec, Sherpa, or Gammapy.


\todo[inline]{To be completed (e.g. ???)}

% mireille : to be discussed
%??? naïve question : what would be the benefit to convert science ready event table data to VOTable?
%Would TOPcat, Aladin, etc. allow more preview steps  , xmatch, multi-wavelength analysis ?


\section{Use Cases}

\subsection{UC1: re-analyse event-list data for a source in a catalog}

After the selection of a source of interest, or a group of sources, one may access different HE data products such as
images, spectra and light-curves, and then want to download the corresponding event-lists and calibrations to further
analyse the data.

%\todo[inline]{To be completed (e.g. Paula, Laurent)}
One of the characteristics of the HE data is that, contrary to what is usually done in optics for example, their optimal
use requires providing users with a view of the processing that generated the data. This implies providing ancillary data,
products with different calibration levels, and possibly linking together products issued by the same processing.
%(LM)


\subsection{UC2: observation preparation}

When planning for new VHE observations, one needs to search for any existing event-list data already available in the
targeted sky regions, and assess if this data is enough to fulfill the science goals.

For this use case, one needs first to obtain the stacked exposure maps of past observations. This quantity is
energy-dependent for VHE data can be derived from pointing position and effective areas that are position- and energy-
dependent associated to each observation.


\subsection{UC3: transient or variable sources}


\todo[inline]{To be completed (e.g. Ada)}


\subsection{UC4: Multi-wavelength and multi-messenger science}

Though there are scientific results based on HE data only, the multi-wavelength and multi-messenger approach is
particularly developed in the HE domain. An astrophysical source of HE radiations is indeed generally radiating
energy in several domains across the electromagnetic spectrum and may be a source of other particles, in particular
neutrino. It is not rare to observe a HE source in radio and to look for counterparts in the infrared, optical or UV
domain and either in X-rays or VHE/UHE band. Spectroscopy is also widely used to identify HE sources.

The HE domain is thus confronted to different kinds of data types and data archives, which leads to interesting use
cases for the development of the VO.

One use case is associated to independent analyses of the multi-wavelength and multi-messenger data. HE data are
analyzed to produce DL5/L3 quantities from DL3/L1 stored in the VO. And the multi-wavelength and multi-messenger
DL5/L3 data stored are retrieved into the VO and associated to realise astrophysical interpretations.

The other growing use case is associated to joint statistical analyses of multi-instrument data at different levels
(DL3/L1 and DL5/L3) by adapted open science analysis tools.

For both use cases, any type of data should be findable on the VO and retrievable. And the data should have a
standardized open format (OGIP, GADF, VODF).

Such use case is already in use with small data sets shared by VHE experiments. In
\citep{2019A&A...625A..10N, 2022A&A...667A..36A}, groups of astronomers working on the Gammapy library had successfully
analyzed DL3 data taken on the Crab nebula by different facilities (MAGIC, H.E.S.S., FACT, VERITAS, Fermi-LAT and HAWC).
A real statistical joint analysis has been performed to derive an emitting model of the Crab pulsar wind nebula over more
than five decades in energy. Such analysis type can be now retrieved in the literacy. It can be found also joint analyses
using X-ray and VHE data \citep{giunti2022}. A proof of concept of joint analysis of VHE gamma-ray and VHE neutrino,
using simulated data, has been also published \citep{unbehaun2024}.

%\todo[inline]{To be completed (e.g. Bruno)}

%
%\subsection{Examples of multi-wavelength analysis}
%
%\subsubsection{Multiple Imaging Atmospheric Cherenkov Telescopes extraction example}
%
%In order to exploit high energy data across a large interval of energy values, and from various IACTs, there is a need
%to harmonise metadata description. Datasets can then be mixed together to create a fused event-list dataset, to expand
%the analysis along the spectral energy axis and study the spectral behaviour of an astronomical object.
%
%This was proposed in \citep{2019A&A...625A..10N} by a group of HE astronomers of various HE facilities.
%%This work used event-list data products as an input from different facilities (MAGIC, H.E.S.S., FACT, VERITAS, etc...).  data for the Crab Nebula computed from the Maximum likelihood functions of each event depending on the IRFs properties.
%In this work, the authors implemented a prototypical data format (GADF) for a small set of MAGIC, VERITAS, FACT, and
%H.E.S.S. Crab nebula observations, and they analyzed them with the open-source Gammapy software package. By combining
%data from Fermi-LAT, and from four of the currently operating imaging atmospheric Cherenkov telescopes, they produced a
%joint maximum likelihood fit of the Crab nebula spectrum.
%
%Such a work has been more recently extended with the HAWC data \citep{2022A&A...667A..36A}, and included neutrino data
%in a common CTA and KM3NeT source search \citep{unbehaun2024}.


\section{IVOA standards of interest for HE}

\subsection{IVOA Recommendations}

\subsubsection{ObsCore and TAP}

Event-list datasets can be described in ObsCore using a dataproduct\_type set to "event". However, this is not widely used in current services, and we observe only a few services with event-list datasets declared in the VO Registry, and mainly the H.E.S.S. public data release (see \ref{sec:hess}).

As services based on the Table Access Protocol \citep{2019ivoa.spec.0927D} and ObsCore are well developed within the VO, it would be a straightforward option to discover HE event-list datasets, as well as multi-wavelength and multi-messenger associated data.

Here is the evaluation of the ObsCore metadata for distributing high energy data set, some features being re-usable as such, and some other features requested for addition or re-interpretation.


\subsubsection{DataLink}

%\todo[inline]{To be completed (e.g. François)} proposed below by FB (2024-01-31)

DataLink specification \citep{2023ivoa.spec.1215B} defines a \{links\} endpoint providing the possibility to link several
access items to each row of the main response table. These links are described and stored in a second
table. In the case of an ObsCore response each dataset can be linked this way (via the via the access\_url
FIELD content) to previews, documentation pages, calibration data as well as to the dataset itself.
Some dynamical links to web services may also be provided. In that case the service input parameters are
described with the help of a "service descriptor" feature as described in the same DataLink specification.

\subsubsection{HiPS}

Several HE observatories are well suited for sky survey, and the Hierarchical Progressive Survey (HiPS) standard is well suited for sky survey exploration. We note that the Fermi facility provides a useful sky survey in the GeV domain.


\subsubsection{MOCs}

Cross-correlation of data with other observations is an important use case in the HE domain. Using the Multi-Order Coverage map (MOC) standard, such operations become more efficient. Distribution of MOCs associated to HE data should thus be encouraged and especially ST-MOCs (space + time coverage)
that make easier the study of transient phenomena.
% (LM)

\subsubsection{MIVOT}

Model Instances in VOTables (MIVOT) defines a syntax to map VOTable data to any model serialized in VO-DML. The annotation operates as a bridge between the data and the model. It associates the column/param metadata from the VOTable to the data model elements (class, attributes, types, etc.) [...]. The data model elements are grouped in an independent annotation block complying with the MIVOT XML syntax. This annotation block is added as an extra resource element at the top of the VOTable result resource. The MIVOT syntax allows to describe a data structure as a hierarchy of classes. It is also able to represent relations and composition between them. It can also build up data model objects by aggregating instances from different tables of the VOTable.
In the case of HE data, this annotation pattern, used together with the MANGO model, will help to make machine-readable quantities that are currently not considered in the VO, such as the hardness ratio, the energy bands, the flags associated with measurements or  extended sources.
% (LM)

\todo[inline]{To be completed}



\subsubsection{Provenance}

Provenance information of VHE data product is a crucial information to provide, especially given the complexity of the data preparation and analysis workflow in the VHE domain. Such complexity comes from the specificities of the VHE data as exposed in sections \ref{sec:vhespec}.

The develoment of the IVOA Provenance Data Model \citep{2020ivoa.spec.0411S} has been conducted with those use cases in mind. The Provenance Data Model proposes to structure this information as activities and entities (as in the W3C PROV recommendation), and adds the concepts of descriptions and configuration of each step, so that the complexity of provenance of VHE data can be exposed.


\todo[inline]{To be completed (e.g. Mathieu)}


\subsection{Data Models in working drafts}

The HE domain and practices could serve as use cases for the developments of data models, such as Dataset DM, Cube DM or MANGO DM.


\section{Topics for discussions in an Interest Group}


\subsection{Definition of a HE event in the VO}
\label{sec:event-bundlle-or-list}

\subsubsection{Current definition in the VO}

The IVOA standards incude the concept of event-list, for example in ObsCore v1.1 \citep{2017ivoa.spec.0509L}, where
event is a dataproduct\_type with the following definition:
\begin{quote}
    \textbf{event}: an event-counting (e.g. X-ray or other high energy) dataset of some sort. Typically this is
    instrumental data, i.e., "event data". An event dataset is often a complex object containing multiple files or
    other substructures. An event dataset may contain data with spatial, spectral, and time information for each
    measured event, although the spectral resolution (energy) is sometimes limited. Event data may be used to produce
    higher level data products such as images or spectra.
\end{quote}

More recently, a new definition was proposed in the product-type vocabulary\footnote{\url{https://www.ivoa.net/rdf/product-type}} (draft):
\begin{quote}
    \textbf{event-list}: a collection of observed events, such as incoming high-energy particles. A row in an event
    list is typically characterised by a spatial position, a time and an energy.
\end{quote}

Such a definition remains vague and general, and could be more specific, including a definition for a HE event, and the
event-list data type.

\subsubsection{Proposed definition to be discussed}

A first point to be discuss would be to converge on a proper definition of HE specific data products:
\begin{itemize}
    \item Propose definitions for a product-type \textbf{event-list}: A collection of observed events, such as incoming
    high-energy particles, where an event is generally characterised by a spatial position, a time and a spectral value
    (e.g. an energy, a channel, a pulse height).
    \item Propose definitions for a product-type \textbf{event-bundle}: An event-bundle dataset is a complex object
    containing an event-list and multiple files or other substructures that are products necessary to analyze the
    event-list. Data in an event-bundle may thus be used to produce higher level data products such as images or spectra.
\end{itemize}

An ObsCore erratum could then propose to change event for event-list and event-bundle.

The precise content of an event-bundle remains to be better defined, and may vary significantly from a facility to another.

For example, Chandra primary products distributed via the Chandra Data Archive include around half a dozen different
types of products necessary to analyze Chandra data (for example, L2 event-list, {\bf PHA spectrum - it is NOT an event
list!}, Aspect solution,
bad pixel map, spacecraft ephemeris, V\&V Report). It is also possible to retrieve secondary products,
containing more products that are needed to recalibrate the data with updated calibrations.

For VHE gamma rays and neutrinos, the DL3 event lists should mandatory be associated to their associated IRFs files. The
links between the event-list and these IRFs should be well defined in the event-bundle.,


\subsection{ObsCore metadata description of an event-list}
\label{sec:obscore}

%%%% texte by Mireille to be checked and merged : start %%
%\include{ObscoreReviewforVOHEcontext_Mireille Louys}
%I have some items to add in the various categories well defined by Mathieu
%%%%%%%%%%texte by Mireille to be merged : end %%

%\subsubsection{Mandatory fields}

\subsubsection{Usage of the mandatory terms in ObsCore}

ObsCore \citep{2017ivoa.spec.0509L} can provide a metadata profile for a data product of type event-list and a qualified access to the distributed file using the Access class from ObsCore (URL, format, file size).

In the ObsCore representation, the event-list data product is described in terms of curation, coverage and access. However, several properties are simply set to NULL following the recommendation: Resolutions, Polarization States, Observable Axis Description, Axes lengths (set to -1)...

We also note that some properties are energy dependent, such as the Spatial Coverage, Spatial Extent, PSF.

\todo[inline]{TODO: show a table with all reused terms , and provide an example}

\begin{itemize}
    \item dataproduct\_subtype = DL3, maybe specific data format (VODF)
    \item calib\_level = between 1 and 2
    \item obs\_collection could contain many details : obs\_type (calib, science), obs\_mode (subarray
configuration), pointing\_mode, tracking\_type, event\_type, event\_cuts, analysis\_type…
    \item s\_ra, s\_dec = maybe telescope pointing coordinates
    \item target\_name : several targets may be in the field of view
    \item s\_fov, s\_region, s\_resolution, em\_resolution... all those values are energy dependent, one should specifiy that the value is at a given energy, or within a range of values.
    \item em\_min, em\_max : add fields expressed in energy (e.g. eV, keV or TeV)
    \item t\_exptime : ontime, livetime, stable time intervals... maybe a T-MOC would help
    \item facility\_name, instrument\_name : minimalist, would be e.g. CTAO and a subarray.
\end{itemize}


\subsubsection{Metadata re-interpretation for the VOHE context}

\paragraph{observation\_id}
In the current definition of ObsCore, the data product collects data from one or several observations. The same happens in HE context.

\paragraph{access\_ref, access\_format}
The initial role of this metadata was to hold the access\_url allowing data access.
Depending on the packaging of the event bundle in one compact format (OGIP, GADF, tar ball, ...)
or as different files available independently in various urls, a datalink pointer can be used for accessing the various parts of IRFs, background maps, etc.
Then in such a case the value for access\_format should be "application/x-votable+xml;content=datalink". The format itself of the data file is then given by the datalink parameter "content-type".
See next section \ref{sec:datalink}.

\paragraph{o\_ucd}
For the even-list table, we can consider all measures stored in columns values have been observed .
The nature of items along time, position and energy axis are identifed in Obscore with ucd as 'time', 'pos.eq.*', 'em.*'
and counted as t\_xel, s\_xel1, s\_xel2, em\_xel which correspond to the number of rows/events candidates observed.

The signal observed is the result of event counting and would be PHA (Pulse height amplitude at detector level) or a number of counts for photons or particles, or a flux, etc.., depending on the data calibration level considered.
ObsCore uses o\_ucd to characterise the nature of the measure.
various UCDs are used for that: o\_ucd=phys.count, phot.count, phot.flux, etc. there is currently no UCD defined for a raw measure like PulseHeightAmplitude, but if needed this can be requested for addition in the UCDList vocabulary. See VEP-UCD-15\_pulseheight.txt proposed at \url{'https://voparis-gitlab.obspm.fr/vespa/ivoa-standards/semantics/vep-ucd/-/blob/master/'}.

Note that these parameters vary between the dataset of calib\_level of 1 (Raw) to the a more advanced data products (calib\_level 2 or 3), which are filtered and rebinned from the original raw event-list.


\subsubsection{Metadata addition required}

\paragraph{ev\_number}
The event list contains a number of rows, representing detections candidates, that have no metadata keyword yet in Obscore.
We propose 'ev\_number' to record this.
In fact the t\_xel, s\_xel1 and s\_xel2, em\_xel elements do not apply for an event list in raw count as it has not been binned yet.

\paragraph{Adding MIME-type to access\_format table}
As seen in section \ref{sec:data_formats} current HE experiments and observatories use their community defined data format for data dissemination.
They encapsulate the event-list table together with ancillary data dedicated to calibration and observing configurations and parameters.
Even if the encapsulation is not standardized between the various projects, it is useful for a client application to rely on the access\_format property in order to send it to an appropriate visualizing tool.

Therefore these can be included in the MIME-type table of ObsCore section 4.7. suggestion for new terms like :
\begin{itemize}
\item application/x-fits-ogip ...
\item application/x-gadf  ...
\item application/x-vodf  ...
\end{itemize}
\todo[inline]{to be completed with proper definition}

\paragraph{energy\_min, energy\_max}
It is not user-friendly for the user to select dataset according to an energy range when the spectral axis is expressed in wavelength and meters. The units and quantities are not familiar to this community.
Moreover the numerical representation of the spectral range in em\_min leads to quantities with many figures and a power as -18 not easily comparable with the current usage.

\todo[inline]{cf. example HESS data shown in Aladin}


\paragraph{t\_gti}

The searching criteria in terms of time coverage require the list of stable/good time intervals to pick appropriate datasets.
t\_min, t\_max is the global time span but t\_gti could contain the list of GTI as a T\_MOC description following the Multi-Order-Coverage (MOC) IVOA standard \citep{2022ivoa.spec.0727F}.
This element could then be compared across data collections to make the data set selection via simple intersection or union operations in T\_MOC representation.
On the data provider's side, the T-MOC element can be computed from the Stable/Good Time Interval table in OGIP or GADF to produce the ObsCore t\_gti field.




\subsubsection{Access and Description of IRFs}

Each IRF file can have an Access object from ObsCore DM to describe a link to the IRF part of the data file.
This can be reflected in an extension of ObsTAP TAP\_SCHEMA.

In the TAP service we could add an IRF Table, with the following columns:

\begin{itemize}
    \item event-list datapublisher\_id
    \item irf\_type, category of response: EffectiveArea, PSF, etc.
    \item irf\_description, one line explanation for the role of the file
    \item Access.url, URL to point to the IRF
    \item Access.format, format of IRF
    \item Access.size, size of IRF file
\end{itemize}



\subsection{Event-list Context Data Model}
\label{sec:EventListContext}

\begin{figure}
\centering
\includegraphics[width=0.9\textwidth]{figures/EventListContext}
\caption{event-list Context Data Model. Notes: STIs and GTIs are slightly different concepts, and multiplicities should be adapted, energy is to specific for an event (intensity?), more products may be attached to a STI/GTI or to IRF.}
\label{fig:EventListContext}
\end{figure}

The event-list concept may include, or may be surrounded by other connected concepts. Indeed, an event-list dataset alone cannot be scientifically analysed without the knowledge of some contextual data and metadata, starting with the good/stable time intervals, and the corresponding IRFs.

The aim of the Event-list Context Data Model is to name and indicate the relations between the event-list and its contextual information. It is presented in Figure~\ref{fig:EventListContext}.


\subsection{Use of Datalink for HE products}
\label{sec:datalink}
There are two options to provide an access to a full event-bundle package.

In the first option, the "event-bundle" dataset (\ref{sec:event-bundlle-or-list}) exposed in the discovery service  contains all the relevant information, e.g. several frames in the FITS file, one corresponding to the event-list itself, and the others providing good/stable time intervals, or any IRF file. This is what was done in the current GADF data format (see \ref{sec:GADF}). In this option, the content of the event-list package should be properly defined in its description: what information is included and where is it in the dataset structure? Obviously the Event-list Context Data Model (see \ref{sec:EventListContext}) would be useful to provide that.

In the second option we provide links to the relevant information from the base "event-list" (\ref{sec:event-bundlle-or-list}) exposed in the discovery service. This could be done using Datalink and a list of links to each contextual information such as the IRFs. The Event-list Context Data Model (see \ref{sec:EventListContext}) would provide the concepts and vocabulary to characterise the IRFs and other information relevant to the analysis of an event-list. These specific concepts and terms describing the various flavors of IRFs and GTI will be given in the semantics and content\_qualifier FIELDS of the DataLink response to qualify the links. The different links can point to different
dereferencable URLs or alternbatively to different fragments of the same drefereencable URL as stated by the DataLink specification.


\todo[inline]{To be completed: show an example ?}



\bibliography{VOHE-Note, ivoatex/docrepo, ivoatex/ivoabib}
%\bibliographystyle{}

\appendix

\section{Changes from Previous Versions}

No previous versions yet.
% these would be subsections "Changes from v. WD-..."
% Use itemize environments.


% NOTE: IVOA recommendations must be cited from docrepo rather than ivoabib
% (REC entries there are for legacy documents only)

\end{document}
